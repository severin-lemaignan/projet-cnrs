\documentclass[a4paper]{article}

\usepackage[T1]{fontenc}
\usepackage[utf8]{inputenc}
\usepackage{graphicx}
\usepackage{fixme}

\usepackage[english, french]{babel}

\usepackage{hyperref}

\usepackage[sorting=none, maxnames=10, doi=true, url=false]{biblatex}
\addbibresource{biblio.bib}
%\AtEveryBibitem{\clearfield{note}}

\usepackage{palatino}

\newcommand{\eg}{{\textit{e.g.~}}}
\newcommand{\etal}{{\textit{et al.~}}}
\newcommand{\ie}{{\textit{i.e.~}}}

\title{Activités scientifiques antérieures}

\author{Séverin Lemaignan}
\date{}

%%% Body
\begin{document}
\maketitle

Ma première contribution à une conférence internationale remonte à 2006, lorsque
j'étais étudiant en master à l'ENSAM/ParisTech. Cet article formalisait une
ontologie pour décrire les processus industriels, et étudiait comment elle
pouvait être appliquée à l'automatisation des lignes de production, via un
paradigme multi-agent~\cite{lemaignan2006mason}. Depuis cette période, mes
activités de recherche se sont focalisées sur cette question: quels ponts bâtir
entre intelligence artificielle (et en particulier, les techniques de
l'ingénierie des connaissances) et interaction homme-robot.

Je propose de présenter ces travaux suivant trois axes : mon travail sur les
outils sémantiques pour la robotique, puis mes recherches sur les liens entre
connaissance et interaction située, puis enfin mes contributions plus récentes
sur la question de la cognition en robotique dans le contexte de l'interaction.

Je terminerai cette première partie en présentant mes autres activités de
support scientifique (enseignement, organisation de colloques et de rencontres
scientifiques, développement logiciels notables).

Toutes les références bibliographiques mentionnées dans cette première partie
sont par ailleurs des références vers des publications dont je suis auteur ou
co-auteur.

\subsection*{Outils sémantiques pour la robotique%
  \label{semantic-tools-for-robotics}%
}

Le point de départ de mes recherches se situe dans l'application de techniques
sémantiques à la robotique. J'ai initialement été introduit aux outils de
l'ingénierie des connaissances durant mon master recherche à l'université Paris
5, que je menais en parallèle de ma formation d'ingénieur à l'ENSAM/ParisTech et
au Karslruhe Institute of Technology (KIT). Mon projet de fin d'étude s'est
intéressé à l'utilisation des ontologies pour le contrôle de systèmes
multi-agents dans un contexte industriel, et a débouché sur une première
publication~\cite{lemaignan2006mason}, depuis citée plus de 80 fois.

J'ai ensuite rejoint pendant un an l'INRIA en tant qu'ingénieur de recherche,
sous la supervision de Michel Parent.  J'ai continué d'y développer, de manière
concrète, l'utilisation d'ontologies pour la robotique, en proposant une
architecture de contrôle de véhicules autonomes basée sur un réseau d'ontologies
dynamiques et partagées entre robots~\cite{mehani2007networking}.

Ces deux premières expériences ont formé le point de départ de mon travail de
doctorat poursuivi conjointement au LAAS-CNRS (2008-2012) sous la supervision de
Rachid Alami, et à l'université technique de Munich (TUM) sous la direction de
Michael Beetz. La problématique initiale qui se posait alors était la suivante :
les robots d'interaction développés dans le laboratoire manipulent des états
internes complexes et dynamiques, mais difficilement observables. L'intuition
poussait à croire qu'en rendant \emph{explicites} les flux de connaissances
entre composants logiciels du robot, non seulement le comportement du robot
serait plus facile à interpréter, mais de plus, de nouvelles opportunités
s'ouvriraient pour les tâches décisionnelles du robot, comme la planification
symbolique ou la supervision du robot.

J'ai donc proposé de concevoir un ``tableau noir'' sémantique à travers lequel
les différents modules décisionnels du robot communiquent, en échangeant des
messages dont la sémantique est explicite. Ce projet, \emph{OpenRobots
Ontology}, est une des contributions importante de mon
doctorat~\cite{lemaignan2010oro}, et représente le fil rouge des projets que j'ai
menés par la suite. L'originalité de l'outil tient à l'introduction d'une
ontologie comme support de représentation des connaissances du robot et des
autres agents (cette technique a depuis été reprise dans de nombreux systèmes de
représentation des connaissances en robotique~\cite{lemaignan2012symbolic}). Cette
ontologie s'appuie sur des concepts de haut niveau dont la sémantique est standard
(issue du projet {\sc OpenCYC}) et non-spécifique à la robotique. Ceci ouvre
en particulier de nouvelles modalités d'introspection et d'interaction avec le
robot, utilisant des concepts plus directement intelligibles par l'homme.

Je discute cet aspect dans~\cite{lemaignan2013explicit}, ainsi que les
principales leçons tirées du déploiement d'une base de connaissances active
comme \emph{OpenRobots Ontology} dans des architectures robotiques complexes
(près de 70 modules logiciels fonctionnant en parallèle lors des expériences
menées avec le robot PR2). L'idée de définir les composants du robot en terme
d'interfaces sémantiques (les connaissances dont il a besoin d'une part, et les
connaissances qu'il produit d'autre part) apparait comme un apport important, de
même que le principe d'utiliser les ontologies pour générer l'ensemble des
inférences \emph{triviales} (du type ``tous les enfants sont aussi des
humains''), essentielles pour que le robot puisse raisonner et interagir dans un
référentiel de connaissances partagé avec l'homme.

Un autre intérêt de cette classe d'outils sémantiques est l'acquisition
automatique et l'intégration par le robot de connaissances issues du web
sémantique. Le projet \emph{OpenRobots Ontology} s'appuie déjà sur des
standards comme {\sc OpenCYC}, et j'ai eu l'occasion d'être invité durant la
conférence INNOROBO 2013 à présenter les pistes permettant d'approfondir les
interactions entre outils sémantiques et robotique.

\subsection*{Connaissances et interaction située%
  \label{semantic-tools-for-grounded-interaction}%
}

Au-delà des recherches sur les outils sémantiques eux-mêmes, mon doctorat visait à
explorer l'intérêt d'une approche sémantique pour l'interaction située, ce qui
inclut les questions d'ancrage symbolique (\emph{symbol grounding}), de
communication, de traduction d'un modèle symbolique en actions, de
représentation des agents interagissant, et en particulier, de la représentation
de leur croyances et intentions.

L'acquisition d'un modèle symbolique du monde et la capacité du robot à prendre
le point de vue des autres agents (\emph{prise de perspective}) a été un des
premiers résultats important~\cite{lemaignan2011anchoring}.  L'utilisation d'une base de
connaissance symbolique permet de représenter le modèle des croyances de chacun
des agents, et ouvre des applications comme l'identification automatique des
propriétés discriminantes des objets de l'environnement du point de vue de
chaque agent~\cite{ros2010which} (travail primé lors de la conférence RoMAN 2010).

Un autre domaine pour lequel l'introduction de flux de connaissances explicites
a été particulièrement fructueuse est celui de la communication naturelle,
située et multi-modale. Parce que le robot manipule en interne des concepts dont
la sémantique est définie et non ambigüe, j'ai montré qu'il est possible alors
d'analyser et de donner du sens à une gamme étendue d'interactions verbales en
langue naturelle~\cite{ros2010robot, lemaignan2011dialogue, lemaignan2013talking}.
J'ai aussi montré que l'intégration d'autres modalités de communication (comme
les gestes) devient transparente, du fait que tous les modules du robot
produisent des connaissances déjà articulées les unes aux autres à travers une
même ontologie générale~\cite{lemaignan2011what, lemaignan2011grounding}.

La couche d'abstraction sémantique que j'ai proposé à enfin ouvert un certain
nombre d'opportunités dans le domaine des actions conjointes homme-robot. Parce
que les croyances, non seulement du robot, mais aussi des agents avec lesquels
le robot interagit, sont représentées de manière explicite, le planificateur
symbolique de tâches peut générer des plans d'action
conjoints~\cite{alami2011when, lemaignan2012bridges, clodic2013robot}. De même, en
maintenant à jour le modèle des croyances de l'homme, il est possible d'influer
de manière symbolique sur l'exécution de ces actions
conjointes~\cite{gharbi2013natural}.

Ces possibilités ont été illustrées dans une série d'expériences menées dans le
cadre du projet européen CHRIS, et dans lesquelles le robot construit un modèle
de l'homme au fil de l'interaction, et l'exploite pour adapter son comportement
durant des séquences de jeu, incluant des inversions de
rôle~\cite{lallee2010towards, lallee2011towards, lallee2012towards}. Ce projet a
été mené en collaboration interdisciplinaire avec l'INSERM et le Max Planck
Institute for Evolutionary Anthropology de Leipzig.

L'ensemble de mon travail de thèse a été recompensé par une mention d'excellence
à l'université de Munich (``Summa Cum Laude'') et le prix de la meilleure
thèse en robotique, décerné par le Groupe de Recherche (GdR) Robotique du CNRS.

\subsection*{Cognition pour l'interaction%
  \label{cognition-for-interaction}%
}

Le travail que j'ai mené sur la représentation et la manipulation de la
connaissance pour l'interaction située a débordé de son périmètre
initial pour s'élargir à la question plus générale de la \emph{cognition pour
l'interaction} chez les robots.

Ce champ de recherche est vaste, et est au c\oe ur du projet de recherche que je
présente dans la seconde partie de ce dossier.

J'ai travaillé sur cette question sous deux angles : une perspective
intégrative d'une part, et une perspective plus exploratoire d'autre part.

J'ai ainsi coordonné l'écriture de deux articles de journaux et d'un chapitre de
livre s'intéressant à l'architecture cognitive du robot dans sa
globalité~\cite{alami2011when, lemaignan2012bridges, lemaignan2015human}. Ils
présentent en particulier comment la manipulation explicite de connaissances
symboliques ouvre des voies nouvelles pour l'intégration des multiples processus
décisionnels au sein d'une architecture robotique complexe.

Dans~\cite{lemaignan2015human}, je présente en particulier les
principaux défis que l'interaction homme-robot pose à l'intelligence
artificielle, en termes d'ancrage, de modèles mentaux, d'attention et
d'action conjointe, d'interaction naturelle multi-modale ou encore d'analyse
spatiale, temporelle et contextualisée de situation. Je montre que ces
questions peuvent être en partie abordées de manière holistique, en s'appuyant
sur des interfaces sémantiques clairement définies.

Parallèlement à cet effort de synthèse au niveau de l'architecture du
robot, j'ai mené plusieurs expériences centrées sur des aspects cognitifs
précis. Ainsi, les expériences de \emph{False Beliefs}~\cite{warnier2012when},
liées à la mise en place d'une théorie de l'esprit chez le robot. Je poursuis
cette ligne de recherche en post-doctorat à l'EPFL, avec en particulier un article
prospectif récemment accepté dans la conférence Human-Robot
Interaction (HRI 2015)~\cite{lemaignan2015mutual}.

Ainsi aussi, le travail que j'ai mené durant mon post-doctorat au LAAS-CNRS
(2012-2013) sur les besoins spécifiques de représentation pour l'interaction.
L'objectif était de concevoir une technique de représentation fusionnant le
\emph{Umwelt} (\emph{monde propre}) spatial et temporel du robot en un modèle
amodal pouvant servir de point d'entrée pour les fonctions cognitives
supérieures du robot. Le prototype que j'ai développé est construit autour de
l'idée de \emph{mondes} que le robot peut manipuler en fonction de son contexte
et de ses besoins.  Chaque monde combine une représentation géométrique et une
\emph{histoire} dans laquelle il peut librement naviguer. Les différents mondes
peuvent évoluer indépendamment, et on peut alors les concevoir comme des mondes
hypothétiques, adaptés par exemple pour simuler et représenter le résultat d'une
planification.

Par ailleurs, je me suis intéressé à la question de la cognition pour
l'interaction sous l'angle complémentaire des mécanismes cognitifs
\emph{humains} en jeu durant une interaction.

J'ai commencé à m'intéresser à cet aspect dans le cadre de la robotique
pédagogique. D'abord de manière expérimentale au sein de l'association Planète
Sciences~\cite{stinckwich2007squeakbot}, puis de manière plus théorique en
suivant un master recherche à l'Université Paris 5 dans le domaine des
Environnements Informatiques pour l'Apprentissage Humain (EIAH), ce qui n'a permis
d'acquérir un certain nombre de bases en interaction homme-machine (HCI), en
particulier en ergonomie.

Ces deux premières expériences ont guidé le choix de mon post-doctorat à l'École
Polytechnique Fédérale de Lausanne (2013-...), où j'ai pris en charge la
coordination des activités robotiques au sein du laboratoire d'ergonomie
éducative. Ce laboratoire (CHILI-EPFL) est connu pour son expertise dans l'étude
des interactions homme-machine et homme-homme durant les phases d'apprentissage.
J'y ai acquis le savoir-faire nécessaire à la menée et l'analyse d'expériences
in-situ (écologiquement valides) impliquant des sujets humains.

J'y supervise plusieurs étudiants, autour de deux projets principaux. Le
premier s'intéresse aux interactions sur la durée entre un robot et des
enfants~\cite{fink2014which}, en s'appuyant sur une plateforme robotique
originale~\cite{mondada2014ranger}. Il m'a permis, entre autres, d'étudier
l'impact des comportements anthropomorphiques sur l'interaction homme-robot sur
le long terme~\cite{lemaignan2014dynamics,lemaignan2014cognitive}, travail qui a
été primé lors de la conférence HRI'2014. Le second projet repose sur l'idée du
\emph{learning by teaching}, et propose de mettre en place une interaction
enfant-robot durant laquelle l'enfant \emph{montre} au robot (Nao) comment
écrire, et par là même développe ses compétences. Ce travail a mis en évidence
une forme d'interaction originale, permettant d'établir une relation pérenne et
soutienant efficacement l'apprentissage. Il a fait l'objet de plusieurs
publications et
communications~\cite{hood2015when,lemaignan2014taught,hood2015cowriter}, et a
récemment été primé par la prestigieuse société américaine d'intelligence
artificielle (AAAI).

\subsection*{Autres activités de support scientifique}

Dans cette dernière section, je propose de parcourir brièvement mes autres
activités scientifiques, afin d'inscrire ma candidature dans le cadre plus large
de l'animation de la vie scientifique.

\subsubsection*{Enseignement et encadrement}

Depuis mon entrée dans la recherche à l'INRIA en 2006, j'ai développé une
activité d'enseignement en parallèle de mon travail de recherche.

J'ai été ainsi assistant d'enseignement en mécatronique à Mines ParisTech
pendant un an (sous la supervision de Bruno Steux), puis, durant mon doctorat,
j'ai été moniteur à l'INSA Toulouse, impliqué en particulier sur les travaux
dirigés et pratiques en Prolog, sur les ontologies, en Java, ADA et SQL.

J'ai aussi organisé et animé de nombreux cours et ateliers dans plusieurs
domaines de la robotique et du développement logiciel. Je suis ainsi intervenu
sur la programmation en robotique avec ROS (\emph{Robot Operating System}), les
techniques de développement collaboratif, les cycles de développement ou encore
l'utilisation d'outils sémantiques comme les ontologies. J'ai organisé plusieurs
tutoriels internationaux sur la simulation en robotique (comme lors de la
conférence EURON, en 2012 au Danemark), et je suis, depuis 2005, formateur au
sein de l'association \emph{Planète Sciences} sur toutes les questions de
robotique pédagogique.

Par ailleurs, j'ai été amené à encadrer plusieurs étudiants ces dernières
années. Trois étudiants au niveau master, dont le travail a débouché sur des
publications~\cite{lemaignan2011grounding,hood2015when,hood2015cowriter}, et,
durant mon post-doctorat à l'EPFL, quatre doctorants : Julia Fink sur la question
de l'anthropomorphisme lors l'interaction homme-machine de longue durée, Shruti
Chandra sur l'utilisation de robots compagnons pour aider les enfants en
difficulté face à l'apprentissage de l'écriture, Ayberk Özgür sur le
développement de nouveaux robots pour l'éducation et Alexis Jacq sur les
questions de robotique cognitive et de modélisation mutuelle.

Au-delà de ce rôle d'encadrement direct, je supervise aussi le groupe robotique
au sein du laboratoire CHILI-EPFL, et j'ai acquis à ce titre une expérience
certaine en gestion d'équipe et de projet (demande de financements, rencontres
de partenaires, etc.).

\subsubsection*{Associations scientifiques et dissémination}

Outre mon implication, déjà mentionnée, dans l'association de diffusion de la
culture scientifique \emph{Planète Sciences}, dont j'ai présidé pendant deux ans
la section robotique, j'ai été, durant mon doctorat, vice-président de
l'association \emph{InCOGnu}, antenne Sud-Ouest de la Fédération Française des
Étudiants et Jeunes Chercheurs en Sciences de la Cognition (FRESCO). Mon
activité au sein de cette association m'a permis de m'ouvrir à la dimension
interdisciplinaire de la recherche en sciences cognitives, en particulier à
travers les conférences mensuelles que nous avons proposés, et où intervenait
des chercheurs confirmés issus de tous les domaines des sciences cognitives.
C'est aussi dans le cadre d'InCOGnu que j'ai participé à l'organisation de la
conférence nationale CJCSC à Toulouse en 2009.

Plus récemment, j'ai organisé et animé en 2012 le premier workshop international
sur le simulateur MORSE auquel une quinzaine de chercheurs européens ont participé.

Je suis par ailleurs relecteur régulier pour les principales conférences en
robotique (IROS, ICRA, RoMAN, HRI).

Parmi les actions que j'ai mené à destination du grand public, la pièce de
théâtre Roboscopie~\cite{lemaignan2012roboscopie} peut être mentionné : à l'automne
2011, j'ai mis en place, en collaboration avec un metteur en scène et un
comédien professionnel, une pièce de théâtre d'une vingtaine de minutes dans
laquelle un robot PR2 donne la réplique à un homme. La pièce interroge sur un
mode métaphorique comment hommes et robots peuvent trouver un espace de vie et
de compréhension mutuels. La pièce a été présentée devant environ 400 personnes
durant le festival La Novela 2011, à Toulouse.

\subsubsection*{Contributions logicielles notables}

Au-delà des développements logiciels directement en lien avec mon travail de
doctorat, et précédemment mentionnés (\emph{OpenRobots
Ontology}~\cite{lemaignan2010oro} et le module d'analyse de la langue naturelle
{\sc Dialogs}~\cite{lemaignan2011grounding}), je me suis impliqué dans le développement
de plusieurs autres projets significatifs dans la communauté robotique.

Le premier de ces projets est le simulateur MORSE~\cite{echeverria2011modular,
echeverria2012simulating}. Le projet a pris une ampleur certaine, avec plus de
25 contributeurs du monde entier, son intégration officielle dans le projet
Debian, et un workshop annuel. J'en suis le concepteur initial, l'un des
principaux contributeurs, et j'anime la communauté des développeurs. J'ai en
particulier travaillé sur l'utilisation de MORSE en interactions
homme-robot~\cite{lemaignan2012morse,lemaignan2014simulation}.

ROS est un autre projet dans lequel je suis impliqué.  J'ai plusieurs
contributions au niveau noyau, et je fais partie de l'équipe qui a adapté ROS
pour le robot Nao.

Enfin, je suis impliqué dans la conception d'outils logiciels plus spécialisés,
comme GenoM~\cite{mallet2010genom3}.

\subsection*{Pour conclure}

Pour conclure cette première partie, je souhaite souligner trois traits de mon
parcours : la place particulière donnée à la validation expérimentale de mes
recherches, la dimension internationale de mon itinéraire, et l'importance de
l'interdisciplinarité.

La table~\ref{experiences} liste les principales expériences que j'ai menées
durant ces dernières années. Comme souvent en robotique, elles sont le résultat
d'un travail d'équipe, mais j'en ai été, pour chacune d'elles, un des
principaux instigateurs.

Toutes ces expériences ont débouché sur des publications, et, ensemble, illustrent la place que
j'accorde à la validation expérimentale de mes recherches. Ces expériences ont
été réalisées sur des robots (et non en simulation), et incluent aussi des
études centrées sur les comportements humains en présence de robots
(\emph{Roboscopie}, \emph{CoWriter}, \emph{Ranger}).

\begin{table*}
\begin{center}

    \begin{tabular}{lp{6cm}l}
\bf{Expérience} & Focus & Ref. \\
\hline
{\it Point \& Learn} (2010) & Acquisition interactive de connaissances &
\cite{lemaignan2010oro} \\
{\it Spy Game} (2010) & Discrimination interactive d'objets & \cite{ros2010which} \\
{\it Moving to London} (2011) & Interaction multi-modale, \newline prise de perspective & \cite{lemaignan2011what} \\
{\it Roboscopie} (2011) & Théâtre, \newline réflexion sur le futur de la HRI & \cite{lemaignan2012roboscopie} \\
{\it Cleaning the table} (2011) & Intégration complète & \cite{alami2011when} \\
{\it I'm in your shoes} (2012) & Fausses croyances & \cite{warnier2012when} \\
{\it Give me this} (2012) & Manipulation naturelle conjointe & \cite{gharbi2013natural} \\
{\it Aperitif time} (2012) & Interaction multi-modale, \newline prise de perspective & \cite{lemaignan2013talking} \\
{\it Ranger} (2013) & Expérience enfant-robot, \newline projections cognitives en HRI & \cite{fink2014which, lemaignan2014dynamics} \\
{\it CoWriter} (2013-2014) & Interaction enfant-robot, education \newline \emph{learning by teaching}, expérience de terrain & \cite{hood2015when,lemaignan2014taught, hood2015cowriter} \\
\hline

\end{tabular}
\end{center}
\caption{Principales expériences menées en interaction homme-robot.}
\label{experiences}
\end{table*}


Le deuxième aspect particulier de mon parcours de jeune chercheur est sa
composante internationale. Dès la période du master, j'ai fait le choix de
suivre un cursus d'ingénieur franco-allemand, qui m'a conduit près de deux ans
en Allemagne, au Karlsruhe Institute of Technology (KIT), et m'a finalement
permis d'obtenir en 2006 la médaille d'or de l'ENSAM/ParisTech. J'ai aussi,
pendant ce master, mené pendant six mois des travaux de recherche en physique
fondamentale au Paul Scherrer Institute en Suisse, ce qui a représenté ma
première immersion à proprement parler dans le monde académique international.

Suite à cela, et bien que cela ne fasse pas partie de mon parcours de chercheur
proprement dit, j'ai voyagé pendant une année autour du monde (2007-2008),
expérience importante tant au niveau personnel qu'au regard de mon ouverture à
l'international.

J'ai souhaité ensuite organiser mon doctorat en co-tutelle, cette
fois avec la Technische Universität de Munich (TUM), ce qui m'a conduit à
nouveau à séjourner plusieurs mois en Allemagne. Mon doctorat allemand, noté, a
reçu la meilleure appréciation possible (\emph{Cum Suma Laude}) dans le système
allemand.

Enfin, le post-doctorat que je réalise actuellement à l'École Polytechnique
Fédérale de Lausanne (EPFL) est la plus récente escale internationale de mon
parcours.

Cet itinéraire international reflète la diversité des environnements
intellectuels dans lesquels j'ai évolué, et me permet aujourd'hui de bien
appréhender le tissu scientifique européen, en particulier dans le domaine de la
robotique cognitive.

Le troisième aspect particulier de mon parcours est sa dimension
interdisciplinaire : depuis la période du master, durant laquelle j'ai suivi en
parallèle une formation d'ingénieur et une formation universitaire dans le
domaine de l'intelligence artificielle appliquée à l'éducation, j'ai tenté de
garder une ouverture sur les questions de cognition humaine. C'est ainsi que je
me suis impliqué dans l'association des jeunes chercheurs en sciences
cognitives, c'est ainsi aussi que j'ai choisi de rejoindre en post-doctorat un
laboratoire centré sur les technologies pour l'éducation, avec un savoir-faire
fort dans le domaine de l'expérimentation homme-machine. J'y ai acquis une
expérience essentielle pour pouvoir aujourd'hui mener des expériences en
interaction homme-robot qui soient pertinentes, écologiquement valides, et
méthodologiquement rigoureuses.

\printbibliography

%%%%%%%%%%%%%%%%%%%%%%%%%%%%%%%%%%%%%%%%%%%%%%%%%%%%%%%%%%%%%%%%%%%%%%%%%%%%%%%%%%%%%%%%%%%%%%%%%%%%%%%
%%%%%%%%%%%%%%%%%%%%%%%%%%%%%%%%%%%%%%%%%%%%%%%%%%%%%%%%%%%%%%%%%%%%%%%%%%%%%%%%%%%%%%%%%%%%%%%%%%%%%%%
%%%%%%%%%%%%%%%%%%%%%%%%%%%%%%%%%%%%%%%%%%%%%%%%%%%%%%%%%%%%%%%%%%%%%%%%%%%%%%%%%%%%%%%%%%%%%%%%%%%%%%%
\end{document}
