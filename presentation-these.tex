\documentclass[a4paper]{article}

\usepackage[T1]{fontenc}
\usepackage[utf8]{inputenc}
\usepackage{graphicx}
\usepackage{fixme}

\usepackage[english, french]{babel}

\usepackage{hyperref}

\usepackage[sorting=ynt, maxnames=10, doi=false, url=false]{biblatex}
\addbibresource{biblio.bib}
\AtEveryBibitem{\clearfield{note}}

\usepackage{palatino}

\newcommand{\eg}{{\textit{e.g.~}}}
\newcommand{\etal}{{\textit{et al.~}}}
\newcommand{\ie}{{\textit{i.e.~}}}

\title{Grounding the Interaction: Knowledge Management for Interactive Robots}

\author{Thèse de Doctorat soutenue par Séverin Lemaignan le 17 juillet 2012}
\date{}

%%% Body
\begin{document}
\maketitle

\section{Résumé}


Avec le développement de la \emph{robotique cognitive}, le besoin d'outils
avancés pour représenter, manipuler, raisonner sur les connaissances acquises
par un robot a clairement été mis en avant. Mais stocker et manipuler des
connaissances requiert tout d'abord d'éclaircir ce que l'on nomme
\emph{connaissance} pour un robot, et comment celle-ci peut-elle être
représentée de manière intelligible pour une machine.

Ce travail s'efforce dans un premier temps d'identifier de manière systématique
les besoins en terme de représentation de connaissance des applications
robotiques modernes, dans le contexte spécifique de la robotique de service et
des interactions homme-robot. Nous proposons une typologie originale des
caractéristiques souhaitables des systèmes de représentation des connaissances,
appuyée sur un état de l'art détaillé des outils existants dans notre
communauté.

Dans un second temps, nous présentons en profondeur ORO, une instanciation
particulière d'un système de représentation et manipulation des connaissances,
conçu et implémenté durant la préparation de cette thèse. Nous détaillons le
fonctionnement interne du système, ainsi que son intégration dans plusieurs
architectures robotiques complètes. Un éclairage particulier est donné sur la
modélisation de la prise de perspective dans le contexte de l'interaction, et de
son interprétation en terme de théorie de l'esprit.

La troisième partie de l'étude porte sur une application importante des
systèmes de représentation des connaissances dans ce contexte de l'interaction
homme-robot : le traitement du dialogue situé. Notre approche et les
algorithmes qui amènent à l'ancrage interactif de la communication verbale non
contrainte sont présentés, suivis de plusieurs expériences menées au
\emph{Laboratoire d'Analyse et d'Architecture des Systèmes} au CNRS à Toulouse,
et au groupe \emph{Intelligent Autonomous System} de l'université technique de
Munich.

Nous concluons cette thèse sur un certain nombre de considérations sur la
viabilité et l'importance d'une gestion explicite des connaissances des agents,
ainsi que par une réflexion sur les éléments encore manquant pour réaliser le
programme d'une robotique \emph{``de niveau humain''}.


\section{Composition du jury}

\subsection{Président du jury}

\textbf{Prof. GHALLAB Malik} ({\tt malik@laas.fr}), HDR, LAAS, CNRS, Toulouse FRANCE


\subsection{Rapporteurs}

\textbf{Prof. SAFFIOTTI Alessandro} ({\tt asaffio@aass.oru.se}), Directeur du \emph{Cognitive Robotic Systems Laboratory}, Université d'Örebro, SUÈDE

\textbf{Prof. HERTZBERG Joachim} ({\tt joachim.hertzberg@uos.de}), Directeur du \emph{Robotics Innovation Center}, DFKI/Université d'Osnabrück, ALLEMAGNE

\subsection{Autres examinateurs}

\textbf{Prof. CHETOUANI Mohamed} ({\tt mohamed.chetouani@upmc.fr}), HDR, ISIR, CNRS, Paris FRANCE

\textbf{ Prof. LEE Dongheui} ({\tt dhlee@tum.de}), TUM, Munich ALLEMAGNE 

\subsection{Directeurs de thèse}

\textbf{Prof. BEETZ Michael} ({\tt beetz@in.tum.de}), Directeur du laboratoire \emph{Intelligent Autonomous Systems}, TUM, Munich ALLEMAGNE 

\textbf{Prof. ALAMI Rachid} ({\tt rachid@laas.fr}), HDR, Directeur du thème robotique, LAAS, CNRS, Toulouse FRANCE 

\end{document}
