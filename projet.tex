\documentclass[a4paper]{article}

\usepackage[T1]{fontenc}
\usepackage[utf8]{inputenc}
\usepackage{graphicx}
\usepackage{fixme}
\usepackage{biblatex}

\usepackage{hyperref}
\addbibresource{biblio.bib}

\newcommand{\eg}{{\textit{e.g.~}}}
\newcommand{\etal}{{\textit{et al.~}}}
\newcommand{\ie}{{\textit{i.e.~}}}


\title{Dossier de candidature CNRS}

\author{}
\date{}

%%% Body
\begin{document}
\maketitle

\section{Activités de recherches passées}
\newrefsection

\subsection{Aperçu}
\label{apercu}

Ma première contribution dans une conférence internationale remonte à 2006,
lorsque j'étais étudiant en master à l'École Nationale des Arts et Métiers
(ENSAM/ParisTech). Cet article formalisait une ontologie pour décrire les
processsus industriels, et étudiait comment elle pouvait être appliquée à
l'automatisation des lignes de production, via un paradigme
multi-agent~\cite{lemaignan2006mason} (cette publication a été citée depuis
plus de 80 fois).

Depuis cette période, mes activités de recherche se sont focalisées sur cette
question: quels ponts bâtir entre intelligence artificielle (et en particulier,
les techniques de l'ingéneurie des connaissances) et interaction homme-robot.

Je propose de présenter ces travaux en adoptant quatre perspectives
complémentaires : tout d'abord, mon travail sur les outils sémantiques pour la
robotique, ensuite, mes recherches sur le lien entre connaissance et
interaction située, puis mes contributions plus récentes sur le lien entre
cognition robotique et interaction, et enfin, les travaux expérimentaux que
j'ai menés ces dernières années.

Je terminerais cette première partie de mon dossier de candidature en
présentant les autres activités de support scientifique (enseignement,
organisation de colloques et de rencontres scientifiques, développement
logiciels notables).

Toutes les références mentionnées dans cette première partie sont des références
vers des publications dont je suis auteur ou co-auteur.

\subsection{Outils sémantiques pour la robotique%
  \label{semantic-tools-for-robotics}%
}

Paris 5?

ENSAM, MASON
\cite{lemaignan2006mason}, over 80 citations

INRIA, semantic V2V networking, control framework
\cite{mehani2007networking}

\cite{Lemaignan2010} ORO

Talk at INNOROBOT2013

\cite{lemaignan2013explicit}


\subsection{Connaissance et interaction située%
  \label{semantic-tools-for-grounded-interaction}%
}

PhD LAAS/TUM

\cite{ros2010which} (ROMAN best paper award)

\cite{Lemaignan2011} (HRI Pioneers)


\subsection{Applications to interaction%
  \label{applications-to-interaction}%
}

Interaction with grounded verbal communication.

\cite{Lemaignan2011a}

\cite{Ros2010a}
\cite{lemaignan2011what}
\cite{lemaignan2011dialogue}
\cite{lemaignan2013talking}

Joint action,...

full paper
The 'CHRIS' suite
\cite{Lallee2010b, Lallee2011, Lallee2012}

workshop
\cite{gharbi2013natural}
\cite{clodic2013on}

\subsection{Cognition pour interaction%
  \label{cognition-for-interaction}%
}

Le travail que j'ai mené sur la représentation et la manipulation de la
connaissance pour l'interaction située a naturellement débordé de son périmètre
initial, pour s'éárgir à la question plus générale de la \emph{cognition chez
les robots pour l'interaction}.

Ce champ de recherche est vaste, et est au c\oe ur du projet de recherche que je
présente dans la seconde partie de ce dossier.

J'ai travaillé sur cette question sous deux angles : une perspective
intégrative, d'une part, et une perspective exploratoire \fxnote{...quel mot}.

J'ai ainsi coordonné l'écriture de plusieurs articles de journaux
\cite{alami2011when, Lemaignan2012, lemaignan2014human} qui présentent comment
la manipulation explicite de connaissances symboliques ouvre des voies nouvelles
pour l'intégration des multiples processus décisionels au sein d'une
architecture robotique complexe.

Dans~\cite{lemaignan2014human} en particulier, je présente en particulier les
principaux défis que l'interaction homme-robot pose à l'intelligence
artificielle, en termes de \emph{grounding}, de modèles mentaux, d'attention et
d'action conjointe, d'interaction naturelle multi-modale ou encore d'analyse
spatiale, temporelle et contextualisée de situation. Je montre aussi que ces
questions peuvent être en partie abordées de manière commune en définissant les
interfaces entre modules décisionels en terme de sémantique échangée.

Parallèlement à cet effort de synthèse au niveau de l'architecture globale du
robot, j'ai mené plusieurs expériences centrées sur des aspects cognitifs
précis. Ainsi, les expériences de \emph{False Beliefs}~\cite{Warnier2012a},
étroitement liées à la mise en place d'une théorie de l'esprit chez le robot.

Ainsi aussi, le travail que j'ai mené durant mon post-doctorat au LAAS-CNRS
(2012-2013) sur les besoins spécifiques de représentations pour l'interaction.
L'objectif était de concevoir une technique de représentation fusionnant le
\emph{Umwelt} (\emph{monde propre}) spatial et temporel du robot en un modèle
amodal pouvant servir de point d'entrée pour les fonctions cognitives
supérieures du robot.

Le prototype que j'ai developpé est construit autour de l'idée de \emph{mondes}
que le robot peut manipuler en fonction de son contexte et de ses besoins.
Chaque monde consiste en une représentation géométrique et une histoire dans
lequel on peut librement naviguer. Les différents mondes peuvent se synchroniser
entre eux, mais aussi évoluer independament (on peut alors les concevoir comme
des mondes hypothétiques, adaptés par exemple pour simuler et représenter le
résultat d'une planification).

Par ailleurs, je me suis aussi intéressé à la question de la cognition pour
l'interaction sous l'angle complémentaire des mécanismes cognitifs
\emph{humains} en jeu durant une interaction.

J'ai commencé à m'interesser à cet aspect dans le cadre de la robotique
pédagogique. D'abord de manière expérimentale au sein de l'association Planète
Sciences~\cite{stinckwich2007squeakbot}, puis de manière plus théorique en
suivant un Master Recherche à l'Université Paris 5 dans le domaine des
Environments Informatiques pour l'Apprentissage Humain (EIAH), ce qui n'a permis
d'acquérir un certain nombre de bases en Human-Computer Interaction (HCI), en
particulier en ergonomie.

Ces deux premières expériences ont facilité choix de mon post-doctorat à l'École
POlytechnique Fédérale de Lausanne, où j'ai pris en charge la coordination des
activités robotiques au sein du laboratoire d'ergonomie éducative. Ce
laboratoire (CHILI-EPFL) est connu pour son expertise dans l'étude des
interactions homme-machine et homme-homme durant les phases d'apprentissage. J'y
ai acquis le savoir-faire nécessaire à la menée et l'analyse d'expériences
in-situ (écologiquement valides) et controllées.

Je supervise deux projets principaux. Le premier s'intéresse aux interactions
sur la durée entre le robot Ranger et des enfants~\cite{fink2014which}. Il m'a
permis, entre autres, d'étudier l'impact des comportements anthropomorphiques
sur l'interaction homme-robot sur le long terme~\cite{lemaignan2014dynamics}. Il
apparait en particulier que les mécanismes cognitifs en jeu chez l'homme lors
d'une interaction suivie dans le temps ont été peu étudié, alors qu'ils sont
critiques pour concevoir et adapter le comportement du robot sur le long terme.

Le second projet repose sur l'idée du \emph{learning by teaching}, et propose de
mettre en place une interaction enfant-robot durant laquelle l'enfant
\emph{montre} au robot (un Nao) comment écrire. La question que l'on se pose est
de savoir si une telle situation peut créer une forme interaction originale qui
permette d'établir une relation pérenne et qui soutienne efficacement
l'apprentissage.

\subsection{Conducting human-robot experiments%
  \label{conducting-human-robot-experiments}%
}

Table \ref{experiences} lists the main experiments I have conducted over the
last 4 years.

\begin{table*}
\begin{center}

\begin{tabular}{lll}
\bf{Expérimence} & Focus & Réference \\
\hline
{\it Point \& Learn} (2010) & Interactive knowledge acquisition & \cite{Lemaignan2010} \\
{\it Spy Game} (2010) & Interactive object discrimination & \cite{ros2010which} \\
{\it Moving to London} (2011) & Multi-modal interaction, perspective taking & \cite{lemaignan2011what} \\
{\it Roboscopie} (2011) & Theater, reflection on the future of HRI & \cite{lemaignan2012roboscopie} \\
{\it Cleaning the table} (2011) & Full stack integration & \cite{alami2011when} \\
{\it I'm in your shoes} (2012) & False beliefs & \cite{Warnier2012a} \\
{\it Give me this} (2012) & Natural joint object manipulation & \cite{gharbi2013natural} \\
{\it Aperitif time} (2012) & Multi-modal interaction, perspective taking & \cite{lemaignan2013talking} \\
{\it CoWriter 1} (2013) & \emph{Child-child experiment}, interaction protocols, field experiment &  \\
{\it Ranger} (2013) & Child-robot experiment, cognitive projections in HRI &  \\
\hline

\end{tabular}
\end{center}
\caption{Main experiments conducted in human-robot interaction.}
\label{experiences}
\end{table*}

postdoc EPFL

\cite{fink2014which}

\subsection{Field experiments%
  \label{field-experiments}%
}


postdoc EPFL



\subsection{Autres activités de support scientifique}

Dans cette dernière section, je propose de parcourir brièvement mes autres
activités scientiques, afin d'inscrire ma candidature dans le cadre plus large
de l'animation de la vie scientifique.

\subsubsection{Associations scientifiques et dissémination}


Vice-chair of InCOGnu, Toulouse local chapter of the national FRESCO Union of Young Reseacher in Cognitive Sciences

Organization of CJCSC

Organization of the first international workshop on MORSE in 2012, with 15 participants from 4 countries.

Roboscopie~\cite{lemaignan2012roboscopie}

\subsubsection{Contributions logicielles notables}

Au-delà des développements logiciels directement en lien avec mon travail de
doctorat, et précédemment mentionnés (ORO~\cite{Lemaignan2010}, {\sc
Dialogs}~\cite{Lemaignan2011a}), je me suis impliqué dans le développement de
plusieurs autres projets dont la portée est significative dans la communauté
robotique.

Le premier de ces projets est le simulateur MORSE~\cite{Echeverria2011,
echeverria2012simulating}. Le projet a pris une ampleur certaine, avec plus de
25 contributeurs du monde entier, son intégration officielle dans le projet
Debian, et un workshop annuel.

J'en suis le concepteur initial, l'un des principaux contributeurs, et le
principal animateur de la communauté des développeurs. J'ai en particulier
travaillé sur l'utilisation de MORSE pour simuler des interactions
homme-robot~\cite{lemaignan2012morse}.

ROS (Robot Operating System) est un autre projet dans lequel je suis impliqué.
J'ai plusieurs contributions dans le noyau de ROS et je me suis aussi fortement
impliqué dans l'adaptation de ROS au robot Nao, en particulier pour faciliter
l'installation de ROS sur ce robot.

Je suis aussi impliqué dans la conception d'outils logiciels plus spécialisés,
comme GenoM~\cite{mallet2010genom3}.

\subsubsection{Enseignement}



Teaching assistant (\emph{moniteur}) at INSA Toulouse for 3 years: Prolog,
ontologies, advanced Java, ADA, advanced databases.

Several tutorials at both local and international levels, on technical topics
ranging from software development techniques (code versioning, building tools)
and programming (Python) to robotics simulation (including a tutorial during EURON2012 conference, Danemark).


\subsection{Conclusion%
  \label{conclusion}%
}


\subsubsection{An International Experience%
  \label{an-international-experience}%
}

ENSAM/KIT

PSI

World tour

LAAS/TUM

EPFL

\printbibliography


%%%%%%%%%%%%%%%%%%%%%%%%%%%%%%%%%%%%%%%%%%%%%%%%%%%%%%%%%%%%%%%%%%%%%%%%%%%%%%%%%%%%%%%%%%%%%%%%%%%%%%%
%%%%%%%%%%%%%%%%%%%%%%%%%%%%%%%%%%%%%%%%%%%%%%%%%%%%%%%%%%%%%%%%%%%%%%%%%%%%%%%%%%%%%%%%%%%%%%%%%%%%%%%
%%%%%%%%%%%%%%%%%%%%%%%%%%%%%%%%%%%%%%%%%%%%%%%%%%%%%%%%%%%%%%%%%%%%%%%%%%%%%%%%%%%%%%%%%%%%%%%%%%%%%%%

\section{Projet de programme de recherche}
\newrefsection

La présentation de mes activités de recherche passées peut être comprise comme
plusieurs regards sur la question de la cognition sociale chez les robots : la
question de la représentation et de la manipulation des connaissances en premier
lieu, la question des pré-supposés à l'interaction naturelle (et verbale en
particulier) ensuite, la question de la cohérence et de la faisabilité d'une
architecture cognitive intégrée, la question de l'évaluation de l'engagement
homme-robot sur le long terme...

Ces différentes facettes sont autant de pièces d'un puzzle auquel je propose de
travailler directement : \emph{comment comprendre l'idée de \emph{cognition
sociale} chez le robot, comment la construire, comment l'évaluer ?}

Ce projet articule plusieurs axes, que j'expose plus en détails ci-après. La
première question que l'on peut poser est : Que nous enseignent les sciences
cognitives \emph{humaines} en terme de compétences cognitives nécessaires à
l'interaction entre agents, et en terme d'évaluation de ces compétences ? Une
synthèse des travaux existants en science humaines, et le travail d'adaptation
à la robotique sont des manques bien identifiés pour avancer sur les questions
d'interaction sociale entre hommes et robots.

L'axe de recherche fondamental que cela ouvre est le travail de compréhension,
de définition et d'application de l'idée de cognition sociale en robotique. Il
s'agit ici d'articuler une problématique aujourd'hui floue car complexe, en
re-pensant l'existant ((neuro-)sciences cognitives, donc, mais aussi
architectures cognitives telle qu'on les connait en intelligence artificielle)
dans le cadre de la robotique sociale, avec ses dimensions incarnées,
sous-spécifiées et dynamiques.

Ensuite un axe de recherche opérationel particulier sur lequel je propose de
travailler est la question de la \emph{représentation} de l'environement
spatial, temporel, social et contextuel du robot. La litérature s'intéresse
essentiellement aux deux premiers points. Il me semble que les deux derniers
sont des objets de recherche peu étudiés et essentiels pour une prise de
décision autonome et complexe dans l'interaction homme-robot.

Enfin, je souhaite inscrire ce programme de recherche dans une perspective
expérimentale forte. La menée systématique d'expériences d'interaction en milieux
écologiquement valides reste rare et sujette à des faiblesses méthodologiques.
En m'appuyant sur l'expérience que j'ai acquise dans ce domaine, je pense
travailler à la mise en place d'un référentiel expérimental dont la méthodologie
est solide, et sur lequel je pourrais m'appuyer pour valider la réflexion sur la
cognition sociale chez le robot.

\subsection{La cognition robotique dans les sciences cognitives}

\subsubsection{Métriques de la cognition en robotique}

L'une des difficultés importantes à laquelle la communauté de la robotique
cognitive fait face est le manque de référentiel permettant d'évaluer (et donc
de comparer et mesurer les progrès) les capacités cognitives de nos robots.
Cela tient à deux raisons principales : d'une part, comme nous venons de le
voir, le périmètre de ce qu'on appelle la robotique cognitive est relativement
mal défini, ce qui rend difficile la création d'une mesure globale des capacités
cognitives ; d'autre part la plupart des outils d'évaluation à notre disposition
viennent de l'étude de la cognition humaine, et sont concrètrement souvent
difficile à appliquer aux robots car la ``hiérarchie'' des compétences
cognitives chez les robots est très différente de ce que l'on rencontre chez
l'homme : les compétences langagières, pour prendre exemple symbolique, sont
difficilement comparable. La production verbale est considérée pour un robot
comme un problème moins difficile (essentiellement traitée par la synthèse
vocale) que la compréhension orale (qui requiert, outre la reconnaissance
vocale, une analyse syntaxique et sémantique). Or les tests de langage que l'on
rencontre en psychologie cognitive vont typiquement tester la capacité à parler
en assumant que l'enfant peut comprendre ce qu'on lui dit.

De fait, la communauté robotique s'appuie aujourd'hui essentiellement sur des
évaluations qualitatives. Langley~\etal\cite{Langley2006} suggère ainsi cinq
dimensions d'évaluation : la \emph{généralité} du système (peut-il s'adapter à
une nouvelle tâche?), la \emph{rationalité} ou pertinence des
inférences/raisonnements/décisions que le système prend, la \emph{réactivité} et
la \emph{persistance} qui évalue si le comportement du système est approprié en
cas de changements non-anticipés, la capacité du système à \emph{ameliorer sa
performance} comme fonction des connaissances supplémentaires qu'on lui fournit,
et enfin, \emph{l'autonomie} du système. On le voit, ces dimensions sont
générales, et peu opérationelles.

Un certain nombre d'outils issus de la psychologie cognitive ont déjà été
explorés de manière concrête sur des robots, comme les expériences d'inversion
de rôle~\cite{Lallee2010b}, de croyances fausses (\emph{False
Beliefs}~\cite{Leslie2000}), lié à la théorie de l'esprit~\cite{Breazeal2006,
Warnier2012a}, ou encore le \emph{Token test}~\cite{DiSimoni1978} pour
l'évaluation de certaines compétences linguistiques~\cite{Mavridis2006}. Ces
expériences ont montré qu'il était possible et utile de ré-utiliser en les
adaptant des tests de psychologie cognitive en robotique. Elles restent
cependant ponctuelles, et une travail plus systématique et plus global reste à
faire. D'autant plus que, comme l'ont récemment souligné
Zhang~\etal\cite{zhang2013evaluation}, les environements et métriques existants
pour l'évaluation des performances cognitives des robots se focalisent pour la
plupart sur des capacités physiques, et ne requièrent pas de capacités avancées
de représentation et manipulation de connaissance.

Eux-même proposent leur une métrique basée sur la mesure du \emph{Fitness to Ideal Model} (FIM) d'un
comportement, qu'ils corrèlent à la \emph{longueur de description} (DLen) de la
commande qui a déclenchée le comportement. L'hypothèse étant que, pour un
comportement donné, plus le robot a des capacités cognitives avancées, plus
courte (c'est à dire, sous-spécifiée) pourra être la commande, le robot inférant
le reste. Cette piste est intéressante, et est à mettre en relation avec
d'autres approches issues des sciences cognitives humaines.

Mon objectif est ici de rechercher et de construire un référentiel solide et
opérationnel pour l'évaluation des compétences cognitives des robots, en
s'intéressant à la fois à l'évaluation de chaque capacité cognitive
indépendente, et à la fois à des métriques globales de l'activité cognitive du
robot. Dans~\cite{lemaignan2013explicit}, je propose une piste à explorer à ce
sujet en définissant la notion de \emph{charge cognitive} du roboten terme de
flux de connaissances à l'intérieur de l'architecture logicielle du robot.

\subsection{La représentation du monde comme fondation de la cognition}

\includegraphics[width=\textwidth]{figs/robots_home_baby_socket.jpg}

\subsection{Un programme expérimental ambitieux}




\printbibliography

\end{document}
